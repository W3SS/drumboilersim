%% Redefinições do modelo da ABNT/padrões do abntex2.
%
%% ==================================================================
%% ==================================================================
%% NÃO EDITAR ESSE ARQUIVO
%% ==================================================================
%% ==================================================================
%


\renewcommand{\familydefault}{\sfdefault}
\renewcommand{\ABNTEXchapterfontsize}{\bfseries\fontsize{12pt}{1em}}
\renewcommand{\ABNTEXsectionfontsize}{\bfseries\fontsize{12pt}{1em}}
\renewcommand{\ABNTEXsubsectionfontsize}{\bfseries\fontsize{12pt}{1em}}
\renewcommand{\ABNTEXsubsubsectionfontsize}{\bfseries\fontsize{12pt}{1em}}
\renewcommand{\normalsize}{\fontsize{12pt}{1em}}
\renewcommand{\arraystretch}{1.2}

\renewcommand{\legend}{\fontsize{11pt}{1em}}


% Secao Primaria (Chapter): Caixa alta, Negrito, tamanho 12
\makeatletter
\settocpreprocessor{chapter}{%
\let\tempf@rtoc\f@rtoc%
\def\f@rtoc{%
\texorpdfstring{\bfseries\MakeTextUppercase{\tempf@rtoc}}{\tempf@rtoc}}%
}
\makeatother

% Margens
\setlrmarginsandblock{3cm}{2cm}{*} % externa / interna
\setulmarginsandblock{3cm}{2cm}{*} % superior / inferior
\checkandfixthelayout%

% Recuo do paragrafo
\setlength{\parindent}{1.25cm} % normativo citatanto o valor de 1,25 qto 1,5cm
% sem espaco entre os paragrafos
\setlength{\parskip}{0cm}

% Espacamentos nos titulos:
% chapter
\setlength{\beforechapskip}{-\onelineskip} %com 0 nao funcionou
\setlength{\afterchapskip}{\onelineskip} % antes do titulo de capitulo
%\setlength\beforechapskip{-18pt}

% section
\setbeforesecskip{\onelineskip}
\setaftersecskip{\onelineskip}

% subsection
\setbeforesubsecskip{\onelineskip}
\setaftersubsecskip{\onelineskip}

% subsubsection
\setbeforesubsubsecskip{\onelineskip}
\setaftersubsubsecskip{\onelineskip}

% subsubsubsection
\setbeforeparaskip{\onelineskip}
\setafterparaskip{\onelineskip}

% espaçamento duplo
\linespread{2}



% ---
% Configurações do pacote backref
% Usado sem a opção hyperpageref de backref
\renewcommand{\backrefpagesname}{Citado na(s) página(s):~}
% Texto padrão antes do número das páginas
\renewcommand{\backref}{}
% Define os textos da citação
\renewcommand*{\backrefalt}[4]{ }
% 	\ifcase #1 %
% 		Nenhuma citação no texto.%
% 	\or
% 		Citado na página #2.%
% 	\else
% 		Citado #1 vezes nas páginas #2.%
% 	\fi}%
% ---

\renewcommand*{\imprimircapa}{
  \begin{capa}
    \center
    \includegraphics[width=.62in]{img/template/logo.png}
    
    \ABNTEXchapterfont Universidade Federal de Uberlândia

    \ABNTEXchapterfont Faculdade de Engenharia Elétrica

    \vfill
    
    {\ABNTEXchapterfont\fontsize{16pt}{1em}\textbf{\imprimirautor}}

    \vfill
    {\ABNTEXchapterfont\fontsize{14pt}{1em}\textbf{\imprimirtitulo}}
    \vfill
    \vspace*{1cm}

    \ABNTEXchapterfont\imprimirlocal%
    
    \ABNTEXchapterfont\imprimirdata%
  \end{capa}
}

\renewcommand{\folhaderostocontent}{
  \begin{center}
    
    \vspace*{4.5cm}
    
    {\ABNTEXchapterfont\fontsize{16pt}{1em}\textbf{\imprimirautor}}

    \vfill
    {\ABNTEXchapterfont\fontsize{14pt}{1em}\textbf{\imprimirtitulo}}
    
    \hspace{.45\textwidth}
    \begin{minipage}{.5\textwidth}
    \SingleSpacing%
    \ABNTEXchapterfont\imprimirpreambulo%

    \vspace*{.5cm}
    Orientador: \imprimirorientador%

    \assinatura{Assinatura do Orientador}
    \end{minipage}%
    \vfill 
    
    \ABNTEXchapterfont\imprimirlocal%
    
    \ABNTEXchapterfont\imprimirdata%

  \end{center}
  
}


% Desenho de duagramas de blocos para sistemas de controle
\tikzstyle{block} = [draw, rectangle, 
    minimum height=3em, minimum width=6em]
\tikzstyle{sum} = [draw, circle, node distance=1cm]
\tikzstyle{input} = [coordinate]
\tikzstyle{output} = [coordinate]
\tikzstyle{pinstyle} = [pin edge={to-,thin,black}]



% Estilos de codigos fonte
\newcommand{\Py}[1]{
  \lstinputlisting[
	inputencoding=latin1,
	language=Python,                % choose the language of the code
	basicstyle=\footnotesize,       % the size of the fonts that are used for the code
	% numbers=left,                   % where to put the line-numbers
	% numberstyle=\footnotesize,      % the size of the fonts that are used for the line-numbers
	% stepnumber=1,                   % the step between two line-numbers. If it is 1 each line will be numbered
	% numbersep=5pt,                  % how far the line-numbers are from the code
	backgroundcolor=\color{white},  % choose the background color. You must add \usepackage{color}
	showspaces=false,               % show spaces adding particular underscores
	showstringspaces=false,         % underline spaces within strings
	showtabs=false,                 % show tabs within strings adding particular underscores
	frame=single,           % adds a frame around the code
	tabsize=4,          % sets default tabsize to 2 spaces
	captionpos=b,           % sets the caption-position to bottom
	breaklines=true,        % sets automatic line breaking
	breakatwhitespace=false,    % sets if automatic breaks should only happen at whitespace
        mathescape=false,
        texcl=false
%	escapeinside={\%*}{*)},          % if you want to add a comment within your code
  ]{\detokenize{srcs/py/#1.py}}
}

