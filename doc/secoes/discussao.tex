Com a interface funcionando através de um MQ, foi possível simplificar
o desenvolvimento de novos aplicativos que interagem com o
simulador. Isso se mostrou verdadeiro ao se desenvolver tanto os
controladores quanto a interface gráfica. Dessa forma, é possível
agora desenvolver vários outros sistemas baseados no simulador, como
interfaces gráficas para fins educacionais, drivers para comunicação
com CLPs reais, simuladores distribuidos e plataformas colaborativas.

Outras formas de se aumentar a confiabilidade do simulador produzido
seria a implementação de modelos reais para todas as válvulas e para
os queimadores, assim como a adição de parâmetros que aparecem em
sistemas reais, como propensão a erros de medição dos transmissores,
grau de interferência externa, dentre outros. Falta ainda a validação
do simulador com dados de caldeiras de maior potência, onde se usa a
circulação forçada.

Outra forma de se extender o trabalho aqui realizado é desenvolvendo
modelos de plantas que consomem o vapor produzido pela caldeira,
podendo então observar o comportamento do simulador em um ambiente
mais próximo da realidade.
