O objetivo do trabalho foi construir um ambiente de simulação que
possa ser usado no ambiente acadêmico e industrial, possibilitando o
maior entendimento de sistemas de caldeiras aquatubulares e o
comportamento da mesma quando em interação com outros equipamentos. O
simulador provou ser de fácil manuseio, além de apresentar uma
facilidade para se desenvolver aplicações que se comunicam com o
mesmo. Portanto, mostrou-se que o simulador é capaz de cumprir seu
propósito.

Também mostrou-se que a utilização do simulador para se validar
projetos de sistemas de controle é muito útil devido à sua
simplicidade. Um projetista de sistemas de controle pode testar seus
algorítmos em um sistema virtual, agilizando o desenvolvimento do
sistema e reduzindo drasticamente os custos e riscos de testes, uma
vez que não é necessária a utilização de uma caldeira real.

Algumas dificuldades impostas pela complexidade do sistema foram
observadas e podem ser foco de estudos mais aprofundados, como a
estimativa de parâmetros construtivos da caldeira e de parâmetros
iniciais usados pelo simulador. Algumas alternativas foram propostas
nesse trabalho, facilitando a implementação do simulador, porém um
método mais rigoroso deve ser testado com dados de plantas reais, a
fim de atestar sua viabilidade.

O projeto foi bem sucedido em seus fins, abrindo espaço para novos
trabalhos no campo de simulação de caldeiras aquatubulares e da
análise de plantas que utilizam esse tipo de gerador de
energia. Várias oportunidades de estudos a partir desse trabalho foram
apresentadas, mostrando que ainda há espaço para discussões sobre o
assunto abordado.
