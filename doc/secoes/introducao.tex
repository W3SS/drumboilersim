O uso de caldeiras aquatubulares é amplamente difundido, sendo comum
em plantas termelétricas ou para geração de energia elétrica em geral
\cite{wikicaldeiras}. No Brasil, o aumento do interesse em fontes
alternativas de energia causado pela crise energética de 2001 resultou
em um aumento da utilização de biomassa como forma de diversificação
da matriz energética do país \cite{dantasusp}. Essa utilização causou
o aumento significativo da utilização de caldeiras de geração de vapor
na indústria sucroalcooleira, onde o resíduo da produção de etanol a
partir da cana de açúcar é usado em sistemas de cogeração de energia
elétrica capazes de suprir toda a demanda eletromecânica e térmica das
usinas e destilarias, operando em Ciclo Rankine com turbinas a vapor
de contrapressão \cite{leme2005}.

Estudos mostram que o nível de água na caldeira é uma das variáveis
mais importantes para o funcionamento correto e seguro de uma
caldeira: em 1995, foi constatado que 30\% das paradas emergenciais de
plantas nucleares do tipo PWR (reator de água pressurizada) na França
foram causadas por uma variação do nível de água do
tubulão \apud{kwatny}{astrom}, mostrando que o controle dessa variável
é extremamente importante.

Um nível de água elevado dificulta a separação entre líquido e vapor,
aumentando a umidade contida no vapor e diminuindo sua qualidade,
podendo chegar a danificar as pás da turbina. Já o baixo nível de água
pode fazer com que a água restante evapore rapidamente em operação com
carga elevada. Essa evaporação rápida pode causar o superaquecimento
da caldeira, podendo queimá-la ou resultar em uma
explosão \cite{yuan}. O superaquecimento da caldeira também aumenta a
necessidade de paradas da planta, além de aumentar os riscos à
segurança \cite{habib}.

O desenvolvimento de sistemas de controle de nível em caldeiras
aquatubulares já foi amplamente estudado, como pode ser visto
em \citeonline{ufrj} e \citeonline{yuan}.  A simulação de uma caldeira
também já foi exaustivamente discutida em outros trabalhos,
como \citeonline{astrom}, \citeonline{adam1999} e \citeonline{habib},
com resultados próximos aos reais \cite{astrom}, porém especificações
detalhadas para implementação de um simulador não estão disponíveis
nesses trabalhos, o que dificulta a utilização prática dos resultados
encontrados.

A partir de modelos dinâmicos do comportamento do nível da caldeira, é
possível projetar um sistema de controle de nível e validá-lo através
do teste em simuladores, como feito em \citeonline{ufrj}. Após o
projeto, um CLP deve ser programado com a lógica desenvolvida e então
inserido na planta. Um possível defeito no CLP só será constatado
durante sua implantação, quando a planta (ou parte dela) está
desligada. Nesses casos, encontrar o problema com o CLP pode ser mais
difícil, pois o técnico estará sob estresse \cite{troubleshootplc},
além de atrasar o reinicio da planta e portanto gerar prejuízos.

Esse trabalho tem como objetivo a implementação de um simulador do
comportamento dinâmico de uma caldeira aquatubular. Espera-se produzir
um sistema de simulação versátil e de fácil configuração. Esse sistema
deverá ter uma interface de comunicação simples, possibilitando a
criação de drivers para diferentes tipos de controladores e sistemas
supervisórios.
